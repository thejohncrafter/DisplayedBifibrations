\documentclass{article}

\usepackage[utf8]{inputenc}
\usepackage[T1]{fontenc}

\usepackage[prologue]{xcolor}
\definecolor[named]{ACMBlue}{cmyk}{1,0.1,0,0.1}
\definecolor[named]{ACMYellow}{cmyk}{0,0.16,1,0}
\definecolor[named]{ACMOrange}{cmyk}{0,0.42,1,0.01}
\definecolor[named]{ACMRed}{cmyk}{0,0.90,0.86,0}
\definecolor[named]{ACMLightBlue}{cmyk}{0.49,0.01,0,0}
\definecolor[named]{ACMGreen}{cmyk}{0.20,0,1,0.19}
\definecolor[named]{ACMPurple}{cmyk}{0.55,1,0,0.15}
\definecolor[named]{ACMDarkBlue}{cmyk}{1,0.58,0,0.21}

\usepackage[colorlinks]{hyperref}
\hypersetup{colorlinks,
  linkcolor=ACMPurple,
  citecolor=ACMPurple,
  urlcolor=ACMDarkBlue,
  filecolor=ACMDarkBlue}

\usepackage{natbib}
\bibliographystyle{plainnat}

\usepackage{listings}
\def\lstlanguagefiles{lstlean.tex}
\def\lstlanguagefiles{lstlean.tex}
\lstset{language=lean}
\definecolor{keywordcolor}{rgb}{0.7, 0.1, 0.1}   % red
\definecolor{tacticcolor}{rgb}{0.0, 0.1, 0.6}    % blue
\definecolor{commentcolor}{rgb}{0.4, 0.4, 0.4}   % grey
\definecolor{symbolcolor}{rgb}{0.0, 0.1, 0.6}    % blue
\definecolor{sortcolor}{rgb}{0.1, 0.5, 0.1}      % green
\definecolor{attributecolor}{rgb}{0.7, 0.1, 0.1} % red

\begin{document}

\title{Displayed Bifibrations}
\author{Julien Marquet-Wagner}
\date{January 2025}

\maketitle

\begin{abstract}
  Displayed Categories are a nicely behaved formulation of a "category over a
  base" that is well-adapted to formalization. We define displayed categories
  in Lean, and use them to study some properties of bifibrations.
\end{abstract}

Displayed Categories were introduced by \citet{dispcats}.

\begin{lstlisting}
structure Quiver : Type (max u v + 1) where
  obj : Type u
  hom : obj → obj → Type v
\end{lstlisting}

\bibliography{bibliography}

\end{document}
